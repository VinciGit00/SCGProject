\documentclass{article}

\usepackage[a4paper,left=18mm,right=18mm,top=20mm,bottom=18mm]{geometry}
\usepackage[italian]{babel}

\usepackage{titling}
\usepackage{graphicx}
\usepackage{subcaption}
\usepackage{float}

\title{Considerazioni sul dataset di SCG }
\author{Ianitchii Alin, Gabriele Marchesi, David Guzman Piedrahita e Marco Vinciguerra}
\date{\today}    

\begin{document}
\maketitle

\section{Partenza}
Si parte dal volume delle vendite (tabella a consuntivo) per valutare l'impiego standard.
\\Facendo un merge (join) succede che la tabella sdoppia troppo tanto.
\\Facendo un group by si può trovare la differenza di volume corretta.
\\Successivamente si può fare un merge per fare il calcolo dei volumi.
\\La differenza è articolo per articolo $\rightarrow$ livello di granularità prodotto 
per prodotto.
\\La differenza ($\delta$) tra budget e consuntivo in termini di pezzi è di 2105


\end{document}
